\documentclass[12pt, a4paper]{exam}
\usepackage{graphicx}
\usepackage[left=0.8in, top=0.7in, total={6.2in,8in}]{geometry}
\usepackage[normalem]{ulem}
\renewcommand\ULthickness{1.0pt}   %%---> For changing thickness of underline
\setlength\ULdepth{1.3ex}%\maxdimen ---> For changing depth of underline

\begin{document}
	%\thispagestyle{empty}
	\noindent
	\begin{minipage}[l]{0.1\textwidth}
		\noindent
		\includegraphics[width=2.4\textwidth]{logo.png}
	\end{minipage}
\hfill
\begin{minipage}[c]{0.8\textwidth}
	\begin{center}
		{\large Universidad La Salle \par
		\small	Ingeniería de Sofware	\par
		\large	Fundamentos de Lenguajes de Programación	\par
\small Karlo Pacha, Susana Mancilla  \par
\small Fecha: Agosto 23, 2022} \par
	\end{center}
\end{minipage}
\par
\vspace{0.2in}
\noindent
\par 
\vspace{0.15in}
\noindent
\centering
{\small \bfseries 	Lea el libro "Programming Language Pragmatics" Scott (2000), Capítulo 1, Sección 1.1 - 1.6
y responda las siguientes preguntas: }

\begin{questions}
	\pointsdroppedatright
	\question ¿Cuál es la diferencia entre lenguaje de maquina y lenguaje ensamblador?
	\begin{parts}
		\part El lenguaje de máquina se ejecuta de forma directa; el lenguaje de ensamblador requiere conversión a código de máquina para ser ejecutado.  \textbf{\droppoints}
	\end{parts}
\vspace{0.2in}



\pointsdroppedatright
	\question ¿En qué circunstancia un lenguaje de alto nivel es superior al lenguaje ensamblador?
	\begin{parts}
		\part Para proyectos grandes, se hace uso de menos instrucciones.  \textbf{\droppoints}
		\vspace{0.05in}
	\end{parts}
\vspace{0.2in}



\pointsdroppedatright
	\question  ¿En qué circunstancia un lenguaje ensamblador es superior al lenguaje de alto nivel?
	\begin{parts}
		\part Cuando se necesite usar o manipular de manera más directa el hardware.  \textbf{\droppoints}
		\vspace{0.05in}
	\end{parts}
\vspace{0.2in}



\pointsdroppedatright
	\question ¿Por qué hay tantos lenguajes de programación?
	\begin{parts}
		\part Por la evolución.  \textbf{\droppoints}
		\vspace{0.05in}
        \part Prósitos especiales.  \textbf{\droppoints}
		\vspace{0.05in}
        \part Preferencia personal.  \textbf{\droppoints}
		\vspace{0.05in}
        \part Expresividad.  \textbf{\droppoints}
		\vspace{0.05in}
        \part Facilidad de uso para los novatos.  \textbf{\droppoints}
		\vspace{0.05in}
        \part Facilidad de implementación.  \textbf{\droppoints}
		\vspace{0.05in}
        \part Fuente abierta.  \textbf{\droppoints}
		\vspace{0.05in}
        \part Excelentes compiladores.  \textbf{\droppoints}
		\vspace{0.05in}
		\part Economía, mecenazgo e inercia.  \textbf{\droppoints}
	\end{parts}
\vspace{0.2in}



\pointsdroppedatright
	\question Nombre 3 lenguajes en las categorias de von Neumann, funcional y orientado a objetos. Dos lenguajes lógicos y 2 concurrentes.
	\begin{parts}
		\part C, Ada, Fortran.  \textbf{\droppoints}
		\vspace{0.05in}
        \part Haskell, ML, Lisp.  \textbf{\droppoints}
		\vspace{0.05in}
        \part Java, Smalltalk, C++.  \textbf{\droppoints}
		\vspace{0.05in}
        \part Prolog, Spreadsheets.  \textbf{\droppoints}
		\vspace{0.05in}
		\part Ada, Modula-3.  \textbf{\droppoints}
	\end{parts}
\vspace{0.2in}



\pointsdroppedatright
	\question ¿Qué distingue a los lenguajes declarativos e imperativos?
	\begin{parts}
		\part Su filosofía, uno sigue la secuencia de pasos indicada y otro va directamente al resultado final esperado.  \textbf{\droppoints}
		\vspace{0.05in}
	\end{parts}
\vspace{0.2in}



\pointsdroppedatright
	\question ¿Cuál es considerado el primer lenguaje de alto nivel?
	\begin{parts}
		\part Fortran.  \textbf{\droppoints}
		\vspace{0.05in}
	\end{parts}
\vspace{0.2in}



\pointsdroppedatright
	\question ¿Cuál es considerado el primer lenguaje funcional?
	\begin{parts}
		\part Lisp.  \textbf{\droppoints}
		\vspace{0.05in}
	\end{parts}
\vspace{0.2in}



\pointsdroppedatright
	\question ¿Por qué los lenguajes concurrentes no estan considerados en la clasificación de Scott (2000) (Figura 1.1).?
	\begin{parts}
		\part Debido a que la distinción entre ejecución concurrente y secuencial es ortogonal a las clasificaciones ya existentes. Los lenguajes "concurrentes" no lo son explicitamente.  \textbf{\droppoints}
		\vspace{0.05in}
	\end{parts}
\vspace{0.2in}



\pointsdroppedatright
	\question Lista las principales fases de un compilador y describe la función de cada fase.
	\begin{parts}
		\part Análisis léxico y sintáctico: Reconocimiento de los elementos, formación y estructura del lenguaje.  \textbf{\droppoints}
		\vspace{0.05in}
        \part Análisis semántico y generación de código intermedio: Revisión de la coherencia de lo ingresado, luego se transforma a código para una máquina abstracta.  \textbf{\droppoints}
		\vspace{0.05in}
        \part Generación de código objetivo: Transformación del código intermedio en código objeto.  \textbf{\droppoints}
		\vspace{0.05in}
		\part Mejora del código: Optimización del código, es una fase opcional. \textbf{\droppoints}
	\end{parts}
\vspace{0.2in}



\pointsdroppedatright
	\question ¿En qué circunstancias tiene sentido que un compilador pase o revise el código varias veces?
	\begin{parts}
		\part Cuando la cantidad de instrucciones son muy altas y es necesario una optimización.  \textbf{\droppoints}
		\vspace{0.05in}
	\end{parts}
\vspace{0.2in}



\pointsdroppedatright
	\question ¿Cuál es el propósito de la tabla de símbolos en un compilador?
	\begin{parts}
		\part La traducción de la entrada.  \textbf{\droppoints}
		\vspace{0.05in}
	\end{parts}
\vspace{0.2in}



	\pointsdroppedatright
	\question ¿En la actualidad, que programa es mas eficiente, uno desarrollado desde cero en ensamblador o uno generado por un compilador?
	\begin{parts}
		\part Uno generado por compilador, actualmente se dispone de recursos suficientes para no tener que limitarnos tanto, añadiendo además que los proyectos son mucho más grandes y hacer algo tan cercano al hardware sería costoso para las personas.  \textbf{\droppoints}
		\vspace{0.05in}
	\end{parts}
\end{questions}
\vspace{0.75in}
\end{document}